\documentclass[
    parspace,
    noindent,
    nohyp,
]{elteiktdk}[2023/04/10]

\usepackage[dvipsnames]{xcolor}
\usepackage{graphicx}
\usepackage{colortbl}
\usepackage{svg}
\usepackage{float}
\usepackage{pgfplots}
\usepackage{arydshln}
\usepackage{fontawesome}
\usepackage{picture,xcolor}
\usepackage{pdflscape}
\usepackage{fancyvrb}
\usepackage[backend=bibtex,style=numeric]{biblatex}

\usepackage{todonotes}

\newcommand{\rhpad}{\vspace{0.6\baselineskip}}

\newcommand{\thesispar}[1]{
\vspace{1em}
\hspace{0.7cm}\parbox[left][][c]{15.8cm}{\linespread{1.2}\selectfont #1}
\vspace{1em}
}

\usepackage[newfloat]{minted}

\title{HoloDB: szintetikus adatok on-the-fly szolgáltatása relációs adatbázisként}
\date{2024}
\author{Horváth Dávid \\ {\small\href{mailto:horvathdown@student.elte.hu}{horvathdown@student.elte.hu}} }
\degree{Programtervező Informatikus BSc}

\supervisor{Dr. Vincellér Zoltán}
\affiliation{Mesteroktató}

\university{Eötvös Loránd Tudományegyetem}
\faculty{Informatikai Kar}
\department{Információs Rendszerek Tanszék}
\city{Budapest}
\logo{elte_cimer_szines}


\addbibresource{references.bib}

\begin{document}

\documentlang{hungarian}

\listoftodos
\cleardoublepage

\makecover
\cleardoublepage

\maketitle

\tableofcontents
\cleardoublepage


\begin{abstract}
ABSTRACT
\end{abstract}



\chapter{STORAGE API}

\section{ARCHITEKTÚRA}

\section{REFERENCIA-IMPLEMENTÁCIÓK}

\subsection{KÉTLÉPÉSES ÉRTÉKKIOSZTÁS}

\subsection{PERMUTÁCIÓK}

A \texttt{Permutation} interfész megvalósításához olyan eljárást keresünk, amellyel változó (és akár óriás) tartományon értelmezett,
seed szerint variálható permutációfüggvény és annak inverze előáll, ahol mindkét függvény hatékony.

\subsubsection{EGYSZERŰ}

Triviális implementációként használhatjuk egyszerűen az azonosságfüggvényt:

$$
P_{id}(i) = i (0 \leq i < n)
$$

Gyorsan számolható, de ránézésre már jó keveredést adó permutációt nyerünk, ha kihasználjuk, hogy az $ai \equiv p \pmod{n}$ lineáris kongruencia
$i$-re és $p$-re nézve egy permutációt ad, ha $a$ és $n$ relatív prímek:

$$
P_n(a,b)(i) = ai + b \mod n
$$

Az $a$ és $b$ számot előre, a seed alapján határozzuk meg.
Az $a$ szám relatív prím kell legyen $n$-re nézve, amit úgy érünk el, hogy ciklusban az euklideszi algoritmussal ellenőrizzük, hogy a legnagyobb közös osztójuk 1-e,
és amíg nem, addig meghívjuk a számot tároló \texttt{LargeInteger} példány \texttt{nextProbablePrime()} metódusát az új jelölt lekéréséhez.
Ez a háttérben a \texttt{BigInteger} osztály ugyanilyen nevű metódusát hívja, amely a Java beépített heurisztikus megoldása a következő valószínű prím megkeresésére.
Nekünk csak egy relatív prím kell, tehát nincs szükség valódi prímtesztre, elég a gyors heurisztika és aztán a legnagyobb közös osztó ellenőrzése.

\subsubsection{FEISTEL}

Az invertálható, változtatható méretű, seed alapján újrakeverhető permutációk könnyen analógiába állíthatók kriptográfiai titkosítófüggvényekkel.
Ezek általában dekódolhatóak, változtatható blokkméretűek és kulcs alapján újrakeverhetőek.
Felmerül a gondolat, hogy közvetlenül egy ilyen kriptográfiai módszert alkalmazzunk a permutáció megvalósításához.
Mindenképpen olyan megoldásra van szükség, amelynél a blokkméret tetszőlegesen megadható, és kellően flexibilis, skálázható.

A Feistel-hálózatok pontosan ilyenek, és emiatt gyakran használják is őket összetett titkosító eljárások keretrendszereként.
A Fesitel-hálózat egy többkörös eljárás, ahol minden körben az adat egyik felének bitjeit módosítjuk.
Ez alapesetben megköveteli a páros blokkméretet, de mindjárt kitérek arra, hogyan lehet ezen enyhíteni.

Az egyes körökben felváltva a bitsorozat bal illetve jobb oldali felét változtatjuk; nevezzük ezt célhelynek, a másik fél tartalmát pedig forrásbiteknek.
A kör abból áll, hogy a forrásbitekre meghívunk egy az adott körhöz dedikált hossztartó hash-függvényt,
és az így kapott eredményt XOR-ral ráfésüljük a célhelyre.
Vegyük észre, hogy ez reverzibilis művelet (bármi legyen is a hash-függvény), ugyanis ha újra alkalmazzuk, az eredeti bitsorozatot kapjuk vissza.
Tehát a Feistel-hálózattal kapott kód visszafejtése úgy történik, hogy a lépéseket visszafelé újra végrehajtjuk.
Általában nem szükséges, hogy az egyes körökhöz különböző hash-függvényt használjunk,
számunkra bőven elég, ha a páros és páratlan körökhöz eltérő függvényt alkalmazunk (szigorúan még ez sem lenne szükséges).

A páros blokkméret követelménye enyhíthető a következőképpen.
Páratlan blokkméret esetén adjunk még egy zéróval feltöltött bitet az adathoz.
Ezzel biztosítjuk, hogy a jobb oldali bitmező is a megfelelő méretű legyen.
Minden olyan kör végén, ahol a jobb felet írtuk felül, az utolsó bitet nullázzuk.
A legvégén pedig csak az utolsó bit nélküli részt adjuk vissza.
Az utolsó bit ekkor nem befolyásolja a számítást, tehát a dekódolás is determinisztikus marad.
Ha egyúttal páros számú kört követelünk meg,
akkor az utolsó kör végére garantáltan ugyanabba a fázisba kerülünk, mint az első kör kezdete előtt;
vagyis a dekódolás folyamata teljesen egyezni fog az elkódolással, ami egyszerűsíti az eljárást.

\todo[inline]{kell egyáltalán a páros számú kör követelménye?}

\todo[inline]{algoritmust csatolni}

Persze, még ha a páros blokkméret követelményét így sikerült is eliminálni,
továbbra is adott a probléma, hogy az $n$ blokkmérethez mindig $2^n$ méretű permutáció tartozik.
Vagyis, ha nem csak 2-hatvány méretű permutációkat akarunk támogatni, méretigazítás szükséges.
Ezt a rejection sampling technika alkalmazásával érjük el.
Az adott $n$ mérethez olyan veszük azt a legkisebb $N$ 2-hatványt, amely $n$-nél nem kisebb.
Amikor egy adott $i$ értékre olyan $p$-t kapunk, amelyre $p \geq n$ ($p < N$),
akkor újra meghívjuk a permutáló függvényt $p$-re.
Addig ismételjük a hívást, míg végre $n$-nél kisebb szám nem születik.

Itt érdemes megjegyezni, hogy más implementációknál a rejection sampling esetleg nem hatékony,
de pontosan ezért vezettük be a $resized(newSize)$ metódust magán a $Permutation$ interfészen.
Így minden implementáció maga döntheti el, hogyan kell átméretezni.
Extrém példa: ha a $P_{+1}(i) = i + 1 \mod N$ permutációt $n$-re kell átméreteznünk,
akkor $i = n$ esetén $N - n$ számú extra iterációt kellene végrehajtani, ami nagy méret esetén rendkívül költséges.
Helyette viszont egyszerűen áttérhetünk modulo $n$-re: $P_{+k}(k)(i) = i + k \mod n$.

\subsubsection{XXXX}

\todo[inline]{egy trükkös saját permutáció, ami elég jó szórást ad (ha sikerül megoldani)}

\subsection{HOSSZKORREKCIÓ (???)}

Hello lorem ipsum.

\pagebreak

\appendix

\phantomsection
\addcontentsline{toc}{chapter}{\biblabel}
\printbibliography[title=\biblabel]
\cleardoublepage

\phantomsection
\addcontentsline{toc}{chapter}{\lstfigurelabel}
\listoffigures
\cleardoublepage

\phantomsection
\addcontentsline{toc}{chapter}{\lsttablelabel}
\listoftables
\cleardoublepage

\chapter{Projektlinkek}

\begin{itemize}
    \item GitHub organization: \par \url{https://github.com/miniconnect/}
    \item Használati példák: \par \url{https://github.com/miniconnect/general-docs/tree/main/examples}
    \item HoloDB projekt: \par \url{https://github.com/miniconnect/holodb}
    \item Docker Hub repository: \par \url{https://hub.docker.com/r/miniconnect/holodb/tags}
    \item \texttt{LargeInteger} benchmark: \par \url{https://github.com/miniconnect/miniconnect-api/tree/master/projects/lang/src/jmh/java/hu/webarticum/miniconnect/lang}
    \item HoloDB benchmark: \par \url{https://github.com/davidsusu/holodb-tdk/tree/main/benchmark}
\end{itemize}

\end{document}
