\documentclass[
    parspace, % Add vertical space between paragraphs
    noindent, % No indentation of first lines in each paragraph
    %nohyp, % No hypenation of words
    %twoside, % Double sided format
    %draft, % Quicker draft compilation without rendering images
    %final, % Set final to hide todos
]{elteiktdk}[2023/10/30]

\usepackage{xcolor}
\newcommand{\guessvalue}{}
%\newcommand{\guessvalue}{\color{red} \textbf{$_{?}$}}

\usepackage{tabularx}
\usepackage{booktabs}
\newcolumntype{Y}{>{\centering\arraybackslash}X}

\usepackage{graphicx}

\usepackage{svg}

\usepackage{todonotes}

\newcommand{\rhpad}{\vspace{0.6\baselineskip}}

% The minted package is also supported for source highlighting
%\usepackage[newfloat]{minted}



\title{TDK-dolgozat címe}
\date{2023}
\author{Horváth Dávid}
\degree{Programtervező Informatikus BSc}
\period{2. évfolyam}

% Superivsor(s)' metadata
\supervisor{Témavezető Tamás}
\affiliation{egyetemi tanársegéd}
\cosupervisor{Témavezető Teréz}
\coaffiliation{egyetemi adjunktus}

\university{Eötvös Loránd Tudományegyetem}
\faculty{Informatikai Kar}
\department{XXXXXXXXXXX Tanszék}
\city{Budapest}
\logo{elte_cimer_szines}

\addbibresource{references.bib}

\begin{document}
\documentlang{hungarian}
%\cleardoublepage


\listoftodos
\cleardoublepage

\makecover
\cleardoublepage

\maketitle

\tableofcontents
\cleardoublepage


\todo[inline, color=blue]{általános célúság, nem csak SQL, hanem egyéb (NoSQL, GraphQL? etc.)}
\todo[inline, color=blue]{deklaratív jelleg, adaptáció különféle tesztekhez}
\todo[inline, color=blue]{értékenként csatolható gyakoriság az eloszláshoz}


\begin{abstract}
Lorem ipsum.
\end{abstract}

\chapter{Motiváció}

\section{Bevezetés}

\todo[inline]{section: Bevezetés}

\begin{itemize}
    \item Relational data as a serverless function
    \item TODO
\end{itemize}

\section{Végtelenített virtuális világok}

\todo[inline]{section: Végtelenített virtuális világok}

\section{A fakerek limitációi}

\todo[inline]{section: A fakerek limitációi}

\section{Néhány visszatérő probléma közös nevezője}

\todo[inline]{section: Néhány visszatérő probléma közös nevezője}

\section{Virtuális mockolás: hézag jelei új termék számára?}

\todo[inline]{section: Virtuális mockolás: hézag jelei új termék számára?}

\chapter{Architektúra}

\section{Felépítés}

\todo[inline]{section: Felépítés}

\begin{figure}
\centering
\includesvg{diagram/simplearch}
\caption{A virtuális adatbázis vázlatos architektúrája}
\label{A virtuális adatbázis vázlatos architektúrája}
\end{figure}

\section{Konfiguráció}

\todo[inline]{section: Konfiguráció}

\section{Kiegészítő eszközök és alkalmazások}

\todo[inline]{section: Kiegészítő eszközök és alkalmazások}

\chapter{A virtuális adattár}

\section{A TreeRandom és kapcsolódó API-k}

\todo[inline]{section: A TreeRandom és kapcsolódó API-k}

\section{A storage API}

\todo[inline]{section: A storage API}

\section{A csak-olvasható alapréteg}

Bármilyen imlementáció legyen is mögötte, a storage API egy relációs adatbázist ír le.
Tehát ahhoz, hogy a virtuális adatok funkcionálisan egy valódi (csak-olvasható) relációs adathalmaz képét adják,
tulajdonképpen nem kell más, mint hogy konzisztens módon elérhetők legyenek a storage API-n keresztül.
A konzisztencia ez esetben két dolgot jelent:

\begin{enumerate}
  \item Felépíthető egy olyan tényleges immutábilis relációs adatbázis ($M$),
        hogy minden lehetséges relációs lekérdezés esetében, amely a közös schemára ($S$) értelmes,
        a virtuális és a tényleges adatbázis esetében visszaadott eredménytábla megegyezik.
  \item Az $S$ schema teljesíti a virtuális adatbázis felhasználói konfigurációját ($C$),
        valamint az $M$-ben szereplő adatok tulajdonságai illeszkednek a $C$-ben leírt megkötésekre.
\end{enumerate}

A virtuális adatokhoz a legalsó szinten a storage API megfelelő hívásaival férünk hozzá.
Ezek a speciális hívások szűk keresztmetszetet képeznek,
hiszen ezek szimulálják például a közvetlen adatelérést is.
Ha ezen hívások performanciája nagyságrendileg (de legalább aszimptotikusan) összemérhető a tényleges adatbázisokéval,
akkor az erre épülő lekérdezésfuttató és egyéb rétegek már a tényleges adatbázisoknál megszokott módon tudnak működni.

A lekérdező műveletek esetében két különösen fontos hozzáférési módot kell kiemelni:

\begin{enumerate}
  \item rekordok véletlen elérése (random access)
  \item adott érték előfordulásainak keresése egy oszlopban (reverse index)
\end{enumerate}

Ha e két hozzáférési mód hatékony, akkor már a lekérdezések jelentős részénél
elérhető a tényleges adatbázisokéval összemérhető performancia.

A storage API-ban úgy definiáltuk a megfelelő interfészt,
hogy a keresésen kívül még néhány további funkciót támogatnia kelljen,
például a rendezést és a $NULL$ értékek kezelését.
A következőkben leírt értékkiosztási módszerek többéségénél természetes módon következik,
hogy ezeket a további elvárásokat is teljesítik.
Az e szempontból problémás esetekre külön ki fogok térni.

\todo[inline]{A csak-olvasható alapréteg további dolgai}

\section{Megoldások a hatékony értékkiosztásra}

Ha a virtuális adatokról beszélünk, elsősorban az egy-egy oszlop alá besorakozó,
közös típussal rendelkező mezőértékekről beszélünk.
Úgy is fogalmazhatunk, hogy az adatokat alapvetően oszlop-orientáltan fogjuk előállítani.
Minden oszlophoz tartozik majd egy virtuális adatlista,
melynek hossza egyenlő az adott oszlopot tartalmazó tábla hosszával,
opcionálisan szerepelhetnek benne $NULL$ értékek,
a többi érték típusának pedig kompatibilisnak kell lennie az oszlophoz megadott típussal.

Bár ez legjobban az oszloponként független adatlistákhoz illeszkedik,
valójában más jellegű értékkiosztási módszer lehet mögötte,
ilyen esetben természetesen csak annyi a teendő,
hogy az egy-egy oszlophoz tartozó listanézetet kell az oszlophoz rendelni.
A több oszlopot érintő megkötések esetében is ez lesz a helyzet.
De lássuk először az egyoszlopos értékkiosztások lehetséges módszereit.

\subsection{Egyoszlopos értékkiosztások}

\todo[inline, color=red]{subsection: Egyoszlopos értékkiosztások (bevezetés)}

\subsubsection{Egyszerű értékkiosztások}

Némely tényleges esetben az oszlopban egészen triviális módon szerepelnek az értékek az oszlop értéklistájában.
Természetesen az ilyen esetek szimulálása a legegyszerűbb.

Triviális eset, ha egy oszlop mezői egy közös konstans értéket tartalmaznak.

Ha olyan adathalmazt szimulálunk, melyre megengedhető a feltevés,
hogy sem törlés, sem releváns értelemben problémás felülírás nem történt még,
úgy egyes oszlopok szekvenciális alapon generálhatók.
Az ilyen esetekben az $n$-edik érték lekérése
egy aritmetikai alapműveletekkel transzformált identitásfüggvénnyel számolható.
Az érték keresése hasonló, az transzformációkat visszafelé alkalmazva megkapjuk az inverz függvényt.

A szekvenciális oszlopoknál az érték megegyezik a rekord 1-től indított sorszámával.

Időbélyegeket is generálhatunk hasonló módon,
amennyiben megengedhető, hogy a szimulált időadatok között egyenletes időközök legyenek.

\subsubsection{Értékkiosztás reguláris kifejezés alapján}

\todo[inline, color=red]{subsubsection: Értékkiosztás reguláris kifejezés alapján}

\subsubsection{Kétfázisú értékkiosztás}

\todo[inline, color=red]{subsubsection: Kétfázisú értékkiosztás}

\subsubsection{Zajosan monoton értékkiosztások}

\todo[inline, color=red]{Zajosan monoton 1}

A fenti monoton-generálás variánsa, nem engedünk üres range-et (elég széles superrange-ek),
és kvázi-random választunk egy végső értéket a range-ből, az alapindex alapján seed-elve.

\todo[inline, color=red]{Zajosan monoton 2}

Warping function with constrained tangent (inverse???) + local noise function with constrained delta.

\begin{figure}
\centering
\includesvg{diagram/distribution}
\caption{A kétfázisú értékkiosztás alapelve: egymás után végrehajtott visszafejthető disztribúció és permutáció}
\label{A kétfázisú értékkiosztás alapelve}
\end{figure}

\begin{figure}
\centering
\includesvg{diagram/getvalue}
\caption{Adatlekérés a kétfázisú értékkiosztásból}
\label{Adatlekérés a kétfázisú értékkiosztásból}
\end{figure}

\begin{figure}
\centering
\includesvg{diagram/findvalue}
\caption{Keresés a kétfázisú értékkiosztásban: a rendezett értékkészlet és a vetítések megfordíthatósága biztosítja a gyors kereshetőséget az virtuális listában}
\label{Keresés a kétfázisú értékkiosztásban}
\end{figure}

\subsubsection{Értékkiosztás oszlopok közötti megkötésekkel}

\todo[inline]{subsubsection: Értékkiosztás oszlopok közötti megkötésekkel}

\subsubsection{Egyéb számított értékek}

\todo[inline]{subsubsection: Egyéb számított értékek}






\section{Az írhatósági réteg}

\todo[inline]{section: Az írhatósági réteg}

\section{Továbbfejlesztési tervek}

\todo[inline]{section: Továbbfejlesztési tervek}

\subsection{Megkötések oszlopok között}

\todo[inline]{subsection: Megkötések oszlopok között}

\chapter{Lekérdezések futtatása}

\section{A beépített SQL-futtató}

\todo[inline]{section: A beépített SQL-futtató}

\section{Használat az Apache Calcite keretrendszerrel}

\todo[inline]{section: Használat az Apache Calcite keretrendszerrel}

\section{Munkamenetek}

\todo[inline]{section: Munkamenetek}

\chapter{Kiegészítő eszközök és alkalmazások}

\section{Hálózati protokollok}

\todo[inline]{section: Hálózati protokollok}

\section{JDBC}

\todo[inline, color=red]{section: JDBC}

\section{REPL}

\todo[inline]{section: REPL}

\todo[inline]{További kiegészítő eszközök?}

\chapter{Összehasonlítás az elterjedt tipikus megoldásokkal}

\section{Megközelítések}

\todo[inline]{section: Megközelítések}

\section{A jelen megoldás sajátos fókuszai}

\todo[inline]{section: A jelen megoldás sajátos fókuszai}

\section{Empirikus tesztek}

\todo[inline]{section: Empirikus tesztek}

\begin{table}
\settowidth\rotheadsize{mmmmmmm}
\begin{tabularx}{\textwidth}{l| c *{8}{|Y} c}
  \toprule
  \parbox{2cm}{Fázis / \\ aspektus} &
    \rothead{MySQL \\ (fájlmásolással)} &
    \rothead{MySQL \\ (SQL dumppal)} &
    \rothead{MySQL + anon. \\ (fájlmásolással)} &
    \rothead{MySQL + anon. \\ (SQL dumppal)} &
    \rothead{MySQL + faker} &
    \rothead{H2 + faker \\ (embedded)} &
    \rothead{H2 Views \\ (embedded)} &
    \rothead{HoloDB \\ (default exec.)} &
    \rothead{HoloDB \\ (Calcite exec.)} 
    \\
  \midrule
  \parbox{2cm}{\rhpad Projekt \\ előfeltétele \rhpad} &
    \multicolumn{2}{c|}{ \begin{tabular}{@{}c@{}}létező \\ adatbázis\end{tabular} } & % MySQL
    \multicolumn{2}{c|}{ \begin{tabular}{@{}c@{}}kész \\ schema\end{tabular} } & % MySQL + anon.
    \multicolumn{5}{c|}{ \begin{tabular}{@{}c@{}}vázlatos \\ schema\end{tabular} } \\ % MySQL + anon.
  \parbox{2cm}{\rhpad Mockolás \\ előfeltétele \rhpad} &
    \multicolumn{2}{c|}{ \begin{tabular}{@{}c@{}}létező \\ adatbázis\end{tabular} } & % MySQL
    \multicolumn{2}{c|}{ \begin{tabular}{@{}c@{}}kész \\ schema\end{tabular} } & % MySQL + anon.
    \multicolumn{3}{c|}{ \begin{tabular}{@{}c@{}}vázlatos \\ schema\end{tabular} } & % MySQL + anon.
    \multicolumn{2}{c}{ \begin{tabular}{@{}c@{}}deklaratív \\ konfiguráció\end{tabular} } \\ % HoloDB
  \parbox{2cm}{\rhpad Data dump \\ (seconds) \rhpad} &
    0.05\guessvalue & % MySQL \\ (fájlmásolással)
    0.94\guessvalue & % MySQL \\ (SQL dumppal)
    0.32\guessvalue & % MySQL + anon. \\ (fájlmásolással)
    1.23\guessvalue & % MySQL + anon. \\ (SQL dumppal)
    0.12\guessvalue & % MySQL + faker
    0.11\guessvalue & % H2 + faker \\ (embedded)
    0 & % H2 Views \\ (embedded)
    0 & % HoloDB \\ (default exec.)
    0 \\ % HoloDB \\ (Calcite exec.)
  \parbox{2cm}{\rhpad Klónozás \\ tesztenként \\ (seconds) \rhpad} &
    \multicolumn{5}{c|}{ \begin{tabular}{@{}c@{}}0.05\guessvalue\end{tabular} } & % MySQL
    0.11\guessvalue & % H2 + faker \\ (embedded)
    0 & % H2 Views \\ (embedded)
    0 & % HoloDB \\ (default exec.)
    0 \\ % HoloDB \\ (Calcite exec.)
  \parbox{2cm}{\rhpad Alap \\ tesztszvit \\ (seconds) \rhpad} &
    \multicolumn{5}{c|}{ \begin{tabular}{@{}c@{}}0.02\guessvalue\end{tabular} } & % MySQL
    0.03\guessvalue & % H2 + faker \\ (embedded)
    0.68\guessvalue & % H2 Views \\ (embedded)
    0.11\guessvalue & % HoloDB \\ (default exec.)
    0.23\guessvalue \\ % HoloDB \\ (Calcite exec.)
  \parbox{2cm}{\rhpad Haladó \\ tesztszvit \\ (seconds) \rhpad} &
    \multicolumn{5}{c|}{ \begin{tabular}{@{}c@{}}1.72\guessvalue\end{tabular} } & % MySQL
    0.96\guessvalue & % H2 + faker \\ (embedded)
    12.7\guessvalue & % H2 Views \\ (embedded)
    3.1\guessvalue & % HoloDB \\ (default exec.)
    3.5\guessvalue \\ % HoloDB \\ (Calcite exec.)
  \parbox{2cm}{\rhpad Tear down \\ (seconds) \rhpad} &
    0.05\guessvalue & % MySQL \\ (fájlmásolással)
    0.94\guessvalue & % MySQL \\ (SQL dumppal)
    0.32\guessvalue & % MySQL + anon. \\ (fájlmásolással)
    1.23\guessvalue & % MySQL + anon. \\ (SQL dumppal)
    0.12\guessvalue & % MySQL + faker
    0.05\guessvalue & % H2 + faker \\ (embedded)
    0.05\guessvalue & % H2 Views \\ (embedded)
    0.01\guessvalue & % HoloDB \\ (default exec.)
    0.01\guessvalue \\ % HoloDB \\ (Calcite exec.)
  \parbox{2cm}{\rhpad Össztárhely \\ (bytes) \rhpad} &
    370M\guessvalue & % MySQL \\ (fájlmásolással)
    510M\guessvalue & % MySQL \\ (SQL dumppal)
    380M\guessvalue & % MySQL + anon. \\ (fájlmásolással)
    520M\guessvalue & % MySQL + anon. \\ (SQL dumppal)
    350M\guessvalue & % MySQL + faker
    360M\guessvalue & % H2 + faker \\ (embedded)
    1M\guessvalue & % H2 Views \\ (embedded)
    1M\guessvalue & % HoloDB \\ (default exec.)
    1M\guessvalue \\ % HoloDB \\ (Calcite exec.)
    
    
  \bottomrule
\end{tabularx}
\caption{Az \texttt{imdb\_ijs} dataset mockolásának performanciája a vizsgált megközelítésekkel, azonos tesztforgatókönyv mellett}
\label{tab:comp_table}
\end{table}


\chapter{Konklúzió}

\todo[inline]{section: Konklúzió}

\end{document}