\documentclass{beamer}
\usetheme{Boadilla}
\usecolortheme{seahorse}
\usepackage{pifont}

\title{On-the-fly relációs adatbázis mockoláshoz}
\author{Horváth Dávid}
\institute{ELTE Informatikai Kar}
\date{2023}

\newcommand{\condpause}{\pause}
%\newcommand{\condpause}{}

\newcommand{\slidetitle}[2]{\frametitle{{\small #1 ~ \ding{226} ~ } #2}}

\begin{document}
\beamertemplatenavigationsymbolsempty

\frame{\titlepage}

\AtBeginSection[]
{
	\begin{frame}
		\frametitle{Tartalom}
		\tableofcontents[currentsection]
	\end{frame}
}

\def\sectiontitle{Motiváció}
\section{\sectiontitle}

\def\motintro{A nem élesben futó szoftvernél az adatbázissal csak a baj van.}
\newif\ifmotfirst
\newcommand{\motcommcont}{
	\item \textbf{Mockolás, fejlesztés} \par ~ ~ csak legyen ott valami, hogy fusson \ifmotfirst \condpause \fi
	\item \textbf{Tesztelés} \par ~ ~ kellenének adatok a tesztfuttatáshoz \ifmotfirst \condpause \fi
	\item \textbf{Kísérletezés, db-tervezés} \par ~ ~ szeretnénk folyamatosan újragondolni \ifmotfirst \condpause \fi
	\item \textbf{Demonstrálás} \par ~ ~ jól jönne egy értelmes fake adathalmaz \ifmotfirst \condpause \fi
	\item \textbf{Oktatás} \par ~ ~ egységes, instant db a tanulóknak \ifmotfirst \condpause \fi
	\item \textbf{Anonimizálás} \par ~ ~ csak valami mást rakjon oda \ifmotfirst \condpause \fi
}
\newcommand{\motaftercont}{
	\item \textbf{Adatbázis-dokumentáció} \par ~ ~ jó lenne látni, mi és miért szerepel a schemában
}


\begin{frame}
\slidetitle{\sectiontitle}{Néhány frusztráló szituáció}

\motintro \smallskip \condpause

\motfirsttrue
\begin{itemize}
	\motcommcont
	\smallskip
	\motaftercont
\end{itemize}

\end{frame}


\begin{frame}
\slidetitle{\sectiontitle}{Néhány frusztráló szituáció}

\motintro \smallskip

\def\motboxpercent{58}
\motfirstfalse
\begin{itemize}
	\smallskip\noindent\makebox[\dimexpr \textwidth * \motboxpercent / 100 \relax][l]{$\left.
		\begin{minipage}{\dimexpr \textwidth * \motboxpercent / 100 \relax}
			\motcommcont
		\end{minipage}
	\right\}$\text{$\rightarrow$ ad hoc kellene egy db}\nulldelimiterspace=0pt}\par\smallskip
    \smallskip
    \motaftercont
\end{itemize}

\end{frame}


\begin{frame}
	\slidetitle{\sectiontitle}{Problémák}
	
	Egy adatbázisszerver ad hoc odateremtése nem egyszerű. \par
	{\small (Általában csak néhány adatra van szükség, de potenciálisan az összesre.)} \smallskip \condpause
	
	\begin{itemize}
		\item \textbf{Definíció} \par ~ ~ definiálni kell, milyen schemát és adatokat szeretnénk látni \condpause
		\item \textbf{Telepítés} \par ~ ~ az adatbázisszerver konfigurálása, rendelkezésre állása nem triviális \condpause
		\item \textbf{Startup} \par ~ ~ el kell indítani az adatbázisszervert (kivéve: beágyazott db) \condpause
		\item \textbf{Schema-építés} \par ~ ~  a schema felépítése is overheadet jelent \condpause
		\item \textbf{Populálás} \par ~ ~ az összes adat generálása és lementése idő- és erőforrásigényes \condpause
		\item \textbf{Memória, tárhely kezelése} \par ~ ~ az adatok sok helyet foglalnak (főleg párhuzamosítva) \condpause
		\item \textbf{Tisztítás} \par ~ ~ a végén törölni kell az elhasznált adatbázist
	\end{itemize}
	
\end{frame}


\begin{frame}
	\slidetitle{\sectiontitle}{Követelmények egy jobb megoldással szemben}
	
	\begin{itemize}
		\item \textbf{Deklaratív konfiguráció} \par ~ ~ {\small barátságos, de nagy kontrollt biztosító konfiguráció lehetősége} \condpause
		\item \textbf{Gyors elindulás} \par ~ ~ {\small lehetőleg közel zero indulási idő} \condpause
		\item \textbf{Kis memóriaigény} \par ~ ~ {\small óriás adatbázis esetén is legyen szolid} \condpause
		\item \textbf{Lekérdezések} \par ~ ~ {\small lehessen SQL lekérdezéseket futtatni (akár többféle nyelvjárással)} \condpause
		\item \textbf{Konzisztencia} \par ~ ~ {\small a független lekérdezések konzisztensek legyenek egymással} \condpause
		\item \textbf{Írási műveletek} \par ~ ~ {\small az adatokat lehessen módosítani (legalább CRUD műveletek)} \condpause
		\item \textbf{Elfogadható run-time sebesség} \par ~ ~ {\small a query-k futtatása is legyen viszonylag gyors} \condpause
		\item \textbf{Skálázhatóság} \par ~ ~ {\small az adatminőség legyen paraméterezhető; akárhány példány futhasson}
	\end{itemize}
	
\end{frame}


\def\sectiontitle{A megoldás váza}
\section{\sectiontitle}


\begin{frame}
	\slidetitle{\sectiontitle}{TODO}
	TODO
\end{frame}


\section{ETC.}
\section{ETC.}
\section{ETC.}

\end{document}
