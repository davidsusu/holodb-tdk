\documentclass[12pt]{article}

\usepackage[utf8]{inputenc}
\usepackage[T1]{fontenc}
\usepackage[magyar]{babel}
\usepackage[a4paper, total={6.7in, 9.3in}]{geometry}
\usepackage{parskip}
\usepackage{hyperref}

\setlength{\parindent}{0pt}

\title{Összefoglaló leírás innOtdk pályázathoz \\ ~ \\ ~ \\ {\Large Az eredeti TDK-pályamunka címe: \\ ~ \\ HoloDB: Relációs demóadatok konzisztens on-the-fly \\ generálása deklaratív konfigurációból} \\ ~}
\author{Horváth Dávid (ELTE-IK) \\ horvathdown@student.elte.hu}
\date{2025. május 15.}

\begin{document}

\begin{titlepage}
\maketitle
\thispagestyle{empty}
\end{titlepage}

\cleardoublepage

A HoloDB egy általános célú adatbázis-szimuláló eszköz, mely egy valószerű adatokkal feltöltött, kereshető relációs adatbázis látszatát képes nyújtani, minimális költségek mellett,
pillanatok alatt betöltve azt egy tömör, deklaratív konfigurációból.
A HoloDB nem pusztán egy újabb fejlesztői eszköz, hanem teljesen új paradigmát kínál a szintetikus adatok terén.

A HoloDB mögötti kutatás arra kereste a választ, hogy lehetséges-e szintetikus adatok egy koherens körét kereshetően is szolgáltatni, azok előre történő, számításigényes generálása (vagy meglévő adatok anonimizálása) és memóriaigényes tárolása nélkül.
Fontos szempont volt, hogy ezt a lehető legáltalánosabb módon, egyúttal elfogadható teljesítmény mellett kell megoldani.
A megvalósítás távoli szakterületeken elért eredményeket ötvöz, a fő nóvum pedig az ezek összeillesztéséből létrejövő újszerű szoftvertípus.
Az alacsony memóriahasználatot úgy érjük el, hogy a szintetikus adatok valójában egyáltalán nincsenek tárolva, hanem on-the-fly kerülnek csak kiszámításra egy-egy lekérdezés közben.
A kereshetőséget virtuális indexek biztosítják, melyeket külön-külön skálázható, kétirányban működő operátorok (például eloszlás, permutáció) kompozíciója valósít meg.
Opcionálisan a táblák fölé helyezhető egy az írási műveleteket is támogató, a módosításokat tömören tároló, átlászó írási réteg;
így az egyébként immutábilis adathalmaz effektíve írhatóvá válik.
Az ötletben rejlő innovációs potenciált az intézményi és az országos TDK-bírálatok is kiemelték, ezért is döntöttem úgy, hogy jelentkezem az innOtdk pályázatra.

A teszteredmények alapján a megoldás működőképes, futásidejű teljesítménye megfelelő, memóriaigénye pedig minimális.
A HoloDB megszünteti a preparálási költségeket, általában jelentősen csökkenti a demóadatokkal való bajlódás miatt fellépő idő- és infrastruktúraigényt, és akár óriás adatbázist is képes szimulálni.
Ez szembemegy a meglévő dogmával, miszerint mindenképpen egy hús-vér adatbázist kell felpopulálni és működtetni annak minden erőforrásigényével együtt, olyan esetben is, amikor egyáltalán nincs szükség valós adatokra.
A HoloDB összességében egy takarékosabb, zöldebb alternatívát kínál.
Az ezt igazoló főbb összehasonlító teszteredmények már bemutatásra kerültek a TDK-n.

A gyakorlatban mindez azt jelenti, hogy különösebb indítási költség nélkül, csekély memóriahasználattal odavarázsolhatjuk az éppen kívánt adathalmazt, például fejlesztéshez, demózáshoz, teszteléshez és egyéb helyzetekben.
A fejlesztőknek, tesztelőknek, prezentálóknak nem kell adatgenerálásra várniuk, sem az érzékeny adatok anonimizálásával törődniük.
Mivel a konfiguráció gyorsan betöltődik, lehetőség van újraindítás nélkül is frissíteni az adathalmazt a konfigurációs fájl módosulásakor, ami különösen jól jön prototipizáláskor, amikor a gyors iterációk az adatok szerkezetét is gyakran érintik (ekkor meglévő, valós adatok még nem is állnak rendelkezésre).
De teljesen más felhasználási lehetőségek is adódnak, a mockadatbázis például leegyszerűsítheti a homokozó adathalmaz rendelkezésre bocsátását oktatási szituációkban.

A HoloDB nyílt forráskódú projektként elérhető ({\small \url{https://github.com/miniconnect/holodb}}).
A meglévő, Java nyelven írt prototípust bárki kipróbálhatja, annak működőképességéről meggyőződhet.
Például Docker konténerként futtatva REST API-n vagy MiniConnect API-n keresztül lehet csatlakozni bármilyen programból, amely ezeket támogatja
(a MiniConnect egy adatbáziselérési API, melyet a JDBC minimalista alternatívájának szántam, illetve a hozzá tartozó kliens-agnosztikus hálózati protokoll).
A HoloDB körül folyamatban van egy gazdag eszközinfrastruktúra kiépítése, magába foglalva többek között a következőket: felhasználóbarát parancssori REPL, NoSQL adapterek (például MongoDB, GraphQL, SPARQL), GraalVM-támogatás, integráció más elterjedt rendszerekkel stb.
Java környezet esetében további lehetőség a JDBC-n keresztüli használat,
a beágyazott elérés, illetve a JPA entitások közvetlen használata konfiguráció helyett.

A megoldás jelen állapotában nagyjából TRL5 szintre tehető:
viszonylag könnyen telepíthető és indítható a legtöbb rendszeren,
az alapfunkciók end-to-end működnek,
és a kényelmi funkciók a nem-szakértő felhasználók számára is lehetővé teszik a használatot.
Azonban világosan mutatkoznak a legfontosabb fejlesztésre váró területek is.

Az innOtdk keretében a következő négy célkitűzést tervezem megvalósítani:

\paragraph{1. Stabil alapok nyílt forráskóddal:}
A nyílt forráskódú alapot minden lényeges szempontból produkciókész szintre kell hozni.
Ehhez minimum szükséges az alap SQL-futtatóban a támogatott SQL-funkciók bővítése,
további teljesítményjavítások,
a konfiguráció bővítése és felülvizsgálata,
valamint helyességi tesztek bevezetése.
A stabilitás legfőbb validációja egy összetettebb valós projekten való tesztelés lenne.

\paragraph{2. Integráció és kompatibilitás:}
Jelenleg a JVM környezetben, valamint a beépített klienseken (például REPL) keresztül való használat támogatott elsődletesen.
Cél, hogy bármilyen programnyelvi környezetből könnyen lehessen csatlakozni a szerverhez,
és hogy hozzáférhető legyen valamely elterjedt grafikus adatbázis-böngészővel (például DBeaver).

\paragraph{3. PaaS platform és monetizáció:}
A cél egy fizetős megoldásként is működtethető \mbox{PaaS} platform kialakítása,
mely HoloDB példányokat futtat,
humán böngészés esetén részben frontendes (WebAssembly-re portolt) számításokkal tehermentesítve a backenden futó konténereket.
Az ehhez tartozó webes felületen a felhasználók a feltöltött (vagy online szerkesztett)
konfigurációt adatbázisként működtethetik, és meg is oszthatják.

\paragraph{4. Nyilvános bemutató:}
A nagyközönségnek szánt bemutatón egy egyszerű de látványos alkalmazáson keresztül kell bemutatni a megoldásban rejlő fő képességeket.
A pulthoz érkező felhasználó két képernyőt kap:
az egyiken a konfigurációt módosíthatja valamilyen könnyített (akár gamifikált) módon,
a másikon pedig az azonnal megváltozó viselkedést próbálhatja ki, mely egy nagy adathalmazon alapszik.
Olyan alkalmazást kell választani, melyen látványosan illusztrálható az óriás kereshető adathalmaz azonnali újrarendeződése,
ami az egyéb létező technológiákkal jelenleg nem megoldható.

Ha a keretek engedik, egyúttal szeretném felmérni a fejlesztők, technológiai vezetők, módszertani szakértők véleményét az ötlet alkalmazhatóságáról a gyakorlatban.
Ez részben kérdőíves formában is történhet, de akár valós projektekben való kipróbálás keretében is, ami összekapcsolható az 1. pontban említett validációval.

\end{document}
