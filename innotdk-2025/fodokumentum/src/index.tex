\documentclass[12pt]{article}

\usepackage[utf8]{inputenc}
\usepackage[T1]{fontenc}
\usepackage[magyar]{babel}
\usepackage[a4paper, total={6.7in, 9.3in}]{geometry}
\usepackage{parskip}

\setlength{\parindent}{0pt}

\title{Összefoglaló leírás innOtdk pályázathoz \\ ~ \\ ~ \\ {\Large Az eredeti TDK-pályamunka címe: \\ ~ \\ HoloDB: Relációs demóadatok konzisztens on-the-fly \\ generálása deklaratív konfigurációból} \\ ~}
\author{Horváth Dávid (ELTE-IK) \\ horvathdown@student.elte.hu}
\date{2025. május 15.}

\begin{document}

\begin{titlepage}
\maketitle
\thispagestyle{empty}
\end{titlepage}

\cleardoublepage

A HoloDB egy Java nyelven írt általános célú adatbázismockoló eszköz, mely egy valószerű adatokkal feltöltött, ténylegesen funkcionáló relációs adatbázist szimulál, pusztán egy tömör, deklaratív konfiguráció alapján.

[konkrétabb motiváció, előzmény, célcsoport, a probléma valós volta, vagy tárolt, vagy nem konzisztens, vagy nem kereshető stb.]

A HoloDB mögötti kutatás arra kereste a választ, hogy lehetséges-e szintetikus adatok egy koherens körét kereshetően is szolgáltatni, azok előre történő, számításigényes generálása (vagy meglévő adatok anonimizálása) és memóriaigényes tárolása nélkül. Fontos szempont volt, hogy ezt a lehető legáltalánosabb módon, egyúttal elfogadható teljesítmény mellett kell megoldani.
A kereshetőséget virtuális indexek biztosítják, amelyek külön-külön skálázható kétirányban működő operátorok (például eloszlás, permutáció) kompozíciója által vannak megvalósítva.
Opcionálisan a táblák fölé helyezhető egy az írási műveleteket is támogató, a módosításokat tömören tároló, átlászó írási réteg.
A teszteredmények alapján a megoldás működőképes, valamint futásidejű teljesítménye is megfelelő.
A megvalósítás távoli szakterületeket ötvöz. Azokban külön-külön is felmutat néhány ötletet, amelyek további kidolgozás után önálló publikálásra lehetnek alkalmasak, a fő nóvum azonban a sok elem összeillesztéséből létrejövő újszerű szoftvertípus.

A HoloDB teljesen megszünteti a preparálási költségeket, általában jelentősen csökkenti a demóadatokkal való bajlódás miatt fellépő idő- és infrastruktúraigényt, és akár óriás adatbázist is képes szimulálni minimális memóriahasználat mellett.
Ez szembemegy azzal a jelenlegi dogmával, hogy szükségszerű lenne egy hús-vér adatbázist populálni és működtetni annak minden erőforrásigényével együtt olyan esetben is, amikor valójában nincs szükség valós adatokra (vagy azok éppen nemkívánatosak, például érzékeny adatok esetén, melyeket minimum anonimizálni kellene),
vagy éppen, mikor azok még nem állnak rendelkezésre.
Az erőforrás-használat csökkenése annál látványosabb, minél nagyobb az adathalmaz, ahogy azt a performanciatesztek is igazolták.
Az erőforráscsökkentéssel kapcsolatos főbb összehasonlító teszteredmények az OTDK-pályamunkában is bemutatásra kerültek.
Konklózióként elmondható, hogy a HoloDB összességében egy takarékosabb, zöldebb alternatívát kínál.

A gyakorlatban mindez azt jelenti, hogy különösebb indítási költség nélkül minimális memóriaigénnyel odavarázsolhatjuk az éppen kívánt adathalmazt, például fejlesztéshez, demózáshoz, teszteléshez és egyéb helyzetekben.
Mivel a konfiguráció gyorsan betöltődik, lehetőség van újraindítás nélkül is frissíteni az adathalmazt a konfigurációs fájl módosulásakor; ez különösen jól jön prototipizáláskor, amikor a gyors iterációk az adatok szerkezetét is gyakran érintik.
De teljesen más felhasználási lehetőségek is adódnak, a mockadatbázis például leegyszerűsítheti a homokozó adathalmaz rendelkezésre bocsátását oktatási szituációkban.
A projekt innovációs potenciálját az intézményi és az országos bírálatok is kiemelték.

A HoloDB nyílt forráskódú projektként elérhető.
A meglévő prototípust bárki kipróbálhatja, annak működőképességéről meggyőződhet.
Például Docker konténerként futtatva REST API-n vagy MiniConnect API-n keresztül lehet csatlakozni bármilyen programból, amely ezeket támogatja
(a MiniConnect egy adatbáziselérési API, melyet a JDBC minimalista alternatívájának szántam, illetve a hozzá tartozó kliens-agnosztikus hálózati protokoll).
A HoloDB körül egy gazdag eszközinfrastruktúra kiépítése zajlik, magába foglalva többek között a következőket: felhasználóbarát parancssori REPL, adatbázisböngészők támogatása (DBeaver), NoSQL adapterek (MongoDB, GraphQL, SPARQL stb.), GraalVM-támogatás stb.
Java környezet esetében további lehetőség a JDBC-n keresztüli használat,
a beágyazott elérés, illetve a JPA entitások közvetlen használata konfiguráció helyett.

A megoldás jelen állapotában nagyjából TRL5 szintre tehető:
viszonylag könnyen telepíthető és indítható a legtöbb rendszeren,
az alapfunkciók end-to-end működnek,
és a kényelmi funkciók a nem-szakértő felhasználók számára is lehetővé teszik a használatot.
Azonban világosan mutatkoznak a legfontosabb fejlesztésre váró területek is.

Az innOtdk keretében a következő négy célkitűzést tervezem megvalósítani:

\paragraph{1. Stabil alapok nyílt forráskóddal:}{
Az nyílt forráskódú alapot minden lényeges szempontból produkciókész szintre kell hozni.
Ehhez minimum szükséges az alapértelmezett a támogatott SQL-funkciók bővítése az alap SQL-futtatóban,
további teljesítményjavítások,
a konfiguráció bővítése, egyértelműsége,
valamint helyességi tesztek bevezetése.
A stabilitás legfőbb validációja egy összetettebb valós projekten való tesztelés lenne.
}

\paragraph{2. Integráció és kompatibilitás:}{
Jelenleg a JVM környezetben, valamint a beépített klienseken (például REPL) keresztül való használat támogatott elsődletesen.
Cél, hogy bármilyen programnyelvi környezetből könnyen lehessen csatlakozni a szerverhez,
és hogy hozzáférhető legyen valamely elterjedt grafikus adatbázis-böngészővel (például DBeaver).
}

\paragraph{3. PaaS platform és monetizáció:}{
A cél egy fizetős megoldásként is működtethető \mbox{PaaS} platform kialakítása,
mely HoloDB példányokat futtat,
humán böngészés esetén részben frontendes (WebAssembly-re portolt) számításokkal tehermentesítve a backenden futó konténereket.
Az ehhez tartozó webes felületen a felhasználók a feltöltött (vagy online szerkesztett)
konfigurációt adatbázisként működtethetik, és meg is oszthatják.
}

\paragraph{4. Nyilvános bemutató:}{
A nagyközönségnek szánt bemutatón egy egyszerű látványos alkalmazáson keresztül kell bemutatni a megoldásban rejlő fő képességeket.
A pulthoz érkező felhasználó két képernyőt kap:
az egyiken a konfigurációt módosíthatja valamilyen könnyített (akár gamifikált) módon,
a másikon pedig az azonnal megváltozó viselkedést próbálhatja ki, mely egy nagy adathalmazon alapszik.
Olyan alkalmazást kell választani, melyen jól illusztrálható az óriás kereshető adathalmaz azonnali újrarendeződése,
ami az egyéb létező technológiákkal jelenleg nem megoldható.
}

\end{document}
