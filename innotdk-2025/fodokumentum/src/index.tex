\documentclass[12pt]{article}

\usepackage[utf8]{inputenc}
\usepackage[T1]{fontenc}
\usepackage[magyar]{babel}
\usepackage[a4paper, total={6.7in, 9.3in}]{geometry}
\usepackage{parskip}
\usepackage{hyperref}
\usepackage{xcolor}

\setlength{\parindent}{0pt}

\title{Összefoglaló leírás innOtdk pályázathoz \\ ~ \\ ~ \\ {\Large A TDK-pályamunka eredeti címe: \\ ~ \\ HoloDB: Relációs demóadatok konzisztens on-the-fly \\ generálása deklaratív konfigurációból} \\ ~}
\author{Horváth Dávid (ELTE-IK) \\ horvathdown@student.elte.hu \\ ~ \\ Témavezető:\\ Dr. Vincellér Zoltán, Mesteroktató (ELTE-IK)}
\date{2025. május 15.}

\begin{document}

\begin{titlepage}
\maketitle
\thispagestyle{empty}
\end{titlepage}

\cleardoublepage

A HoloDB egy általános célú adatbázis-szimuláló eszköz, mely képes valószerű adatokkal feltöltött, kereshető relációs adatbázis látszatát kelteni
minimális költségek mellett, pillanatok alatt betöltve azt egy tömör, deklaratív konfigurációból.
Nem pusztán egy újabb fejlesztői eszközről van szó, a HoloDB a szintetikus adatok kezelésében jelentősen új megközelítést kínál.

A HoloDB mögötti kutatás arra kereste a választ, hogy lehetséges-e szintetikus adatok egy koherens körét kereshetően is szolgáltatni, mégpedig azok előre történő, számításigényes legenerálása és memóriaigényes tárolása nélkül.
Fontos szempont volt, hogy ezt a lehető legáltalánosabb módon, egyúttal elfogadható teljesítmény mellett kell megoldani.
A megvalósítás távoli szakterületeken elért eredményeket ötvöz, a fő nóvum pedig az ezekből összeálló újszerű szoftvertípus.
A különösen alacsony memóriahasználatot az teszi lehetővé, hogy a szintetikus adatok valójában egyáltalán nincsenek tárolva, hanem on-the-fly kerülnek csak kiszámításra egy-egy lekérdezés kiszolgálása közben.
A kereshetőséget teljesen virtuális indexek biztosítják, melyeket külön-külön skálázható, kétirányban működő operátorok (például eloszlás, permutáció) kompozíciói valósítanak meg.
Opcionálisan a táblák fölé helyezhető egy az írási műveleteket is támogató, a módosításokat tömören tároló írási réteg;
így az egyébként immutábilis adathalmaz a kliens felé már effektíve írhatónak mutatkozik.
Az ötletben rejlő innovációs potenciált az intézményi és az országos TDK-bírálatok is kiemelték; ez is motivált abban, hogy a pályamunkát az innOtdk pályázatra továbbvigyem.

A teszteredmények alapján a megoldás működőképes, futásidejű teljesítménye megfelelő, memóriaigénye minimális, a preparálási költségeket pedig egészében megszünteti.
Összességében jelentősen csökkenti a demóadatokkal való bajlódás miatt fellépő idő- és infrastruktúraigényt, és akár óriás adatbázist is képes szimulálni.
Ez szembemegy a meglévő dogmával, miszerint mindenképpen egy hús-vér adatbázist kell felpopulálni és működtetni annak minden erőforrásigényével együtt, olyan esetben is, amikor egyáltalán nincs szükség valós adatokra.
A HoloDB végeredményben egy takarékosabb, zöldebb alternatíva.
Az ezt igazoló főbb összehasonlító teszteredmények bemutatásra kerültek a TDK-n.

A gyakorlatban mindez azt jelenti, hogy különösebb indítási költség nélkül, csekély memóriahasználattal odavarázsolhatjuk az éppen kívánt adathalmazt, például fejlesztéshez, teszteléshez, demózáshoz és egyéb helyzetekben.
A fejlesztőknek, tesztelőknek, prezentálóknak nem kell adatgenerálásra várniuk, sem az érzékeny adatok anonimizálásával bajlódniuk (amennyiben valós kiinduló adatok egyáltalán rendelkezésre állnak).
Mivel a konfiguráció gyorsan betöltődik, lehetőség van újraindítás nélkül is frissíteni az adathalmazt a konfigurációs fájl módosulásakor, ami különösen jól jön prototipizáláskor, hiszen ekkor a gyors iterációk jellemzően az adatmodellt is érintik.
De teljesen más felhasználási lehetőségek is adódnak, a deklaratív adatbázis például leegyszerűsítheti a homokozó adathalmaz rendelkezésre bocsátását oktatási szituációkban.

A HoloDB nyílt forráskódú projektként elérhető ({\small \url{https://github.com/miniconnect/holodb}}).
A meglévő, Java nyelven írt prototípust bárki kipróbálhatja, annak működőképességéről meggyőződhet.
Például Docker konténerként futtatva REST API-n vagy MiniConnect API-n keresztül lehet csatlakozni bármilyen programból, amely ezeket támogatja
(a MiniConnect: egy adatbázis-elérési API, melyet a JDBC minimalista alternatívájának szántam, illetve a hozzá tartozó kliens-agnosztikus hálózati protokoll).
A HoloDB körül folyamatban van egy kiterjedtebb eszközinfrastruktúra kiépítése, magába foglalva többek között a következőket: felhasználóbarát parancssori REPL, NoSQL adapterek (például MongoDB, GraphQL, SPARQL), GraalVM-es változat, integráció más rendszerekkel.
Java környezet esetében további lehetőség a JDBC-n keresztüli használat,
a beágyazott elérés, illetve a JPA entitások közvetlen használata konfiguráció helyett.

A megoldás jelen állapotában nagyjából TRL5 szintre tehető:
viszonylag könnyen telepíthető és indítható a legtöbb rendszeren,
az alapfunkciók end-to-end működnek,
és a kényelmi funkciók a nem-szakértő felhasználók számára is lehetővé teszik a használatot.
Azonban világosan mutatkoznak a legfontosabb fejlesztésre váró területek is.

Az innOtdk keretében a következő négy célkitűzést tervezem megvalósítani:

\paragraph{1. Stabil alapok nyílt forráskóddal:}
A nyílt forráskódú komponenseket minden lényeges szempontból produkciókész szintre kell hozni ahhoz, hogy egy robusztus infrastruktúra alapját képezhessék.
Ehhez szükséges az alapértelmezett lekérdezésfuttatóban a támogatott SQL-funkciók bővítése,
további teljesítményjavítások,
a konfiguráció bővítése és felülvizsgálata,
valamint end-to-end helyességi tesztek bevezetése.

\paragraph{2. Integráció és kompatibilitás:}
Jelenleg elsősorban a JVM környezetben, valamint a beépített klienseken (például REPL) keresztül való használat tesztelt és támogatott.
Cél, hogy bármilyen programnyelvi környezetből könnyen lehessen csatlakozni a szerverhez,
és hogy hozzáférhető legyen valamely elterjedt grafikus adatbázis-böngészőből (például DBeaver).

\paragraph{3. PaaS platform és monetizáció:}
A cél egy fizetős megoldásként is működtethető \mbox{PaaS} szolgáltatás kialakítása,
mely HoloDB példányokat futtat,
humán böngészés esetén részben frontendes (WebAssembly-re portolt) számításokkal tehermentesítve a backenden futó konténereket.
Az ehhez tartozó webes felületen a felhasználók a feltöltött (vagy online szerkesztett)
konfigurációt adatbázisként működtethetik, és meg is oszthatják.

\paragraph{4. Nyilvános bemutató:}
Az on-site bemutatón egy egyszerű de látványos alkalmazáson keresztül lenne célszerű bemutatni a megoldás erősségeit.
A tervezett szcenárióban a pulthoz érkező érdeklődő két képernyőt kap:
az egyiken a konfigurációt módosíthatja valamilyen könnyített (akár gamifikált) módon,
a másikon pedig az azonnal megváltozó viselkedést próbálhatja ki, mely egy nagy adathalmazon alapszik.
Olyan alkalmazást kell választani, melyen látványosan illusztrálható a kereshető óriás adathalmaz azonnali újrarendeződése,
ami az egyéb létező technológiákkal jelenleg nem megoldható.

Továbbá párhuzamosan, amennyire a keretek engedik,
szeretném felmérni a fejlesztők, technológiai vezetők, módszertani szakértők véleményét az ötlet alkalmazhatóságáról a gyakorlatban.
Ez a kérdőíves formától egészen a valós projektekben való kipróbálásig terjedhet.

\end{document}
